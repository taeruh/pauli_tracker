\KOMAoptions{
  headlines=2.1,
  captions=tableheading,
  DIV=17,
  toc=bib,
  toc=listof,
  toc=sectionentrywithdots,
  listof=flat,  
  headings=optiontohead,
}
\addtokomafont{disposition}{\mdseries\slshape}
\addtokomafont{sectionentry}{\rmfamily\mdseries\normalfont}
\addtokomafont{section}{\Large}
\addtokomafont{subsection}{\Large}
\addtokomafont{subsubsection}{\large}
\addtokomafont{descriptionlabel}{\ttfamily}
\addtokomafont{footnote}{\slshape}
\addtokomafont{footnotelabel}{\bfseries}
\addtokomafont{footnotereference}{\bfseries}
\addtokomafont{itemizelabel}{\bfseries}
\addtokomafont{labelitemi}{\bfseries}
\addtokomafont{labelinglabel}{\bfseries}
\addtokomafont{labelingseparator}{\rmfamily}
\addtokomafont{pageheadfoot}{\slshape}
\addtokomafont{caption}{\slshape}
\addtokomafont{captionlabel}{\upshape\bfseries}
\RedeclareSectionCommand[tocnumwidth=1.30em]{section}
\RedeclareSectionCommand[tocnumwidth=2.05em, tocindent=1.3em]{subsection}
\RedeclareSectionCommand[tocnumwidth=2.80em, tocindent=3.35em]{subsubsection}
\DeclareTOCStyleEntry[beforeskip=3pt plus .2pt]{section}{section}
\setcounter{secnumdepth}{3}
\setcounter{tocdepth}{3}
\RequirePackage{setspace}
\setstretch{1.065}
\setlength{\parindent}{0.4cm}
\setlength{\parskip}{0.07cm plus 0.013cm minus 0.06cm}
\RequirePackage[autooneside=false]{scrlayer-scrpage}
\pagestyle{scrheadings}
\clearpairofpagestyles
\addtokomafont{pageheadfoot}{\small}
\renewcommand*{\sectionmarkformat}{}
\automark[subsection]{section}
\automark*[subsubsection]{}
\ihead{\leftmark}
\ohead{\ifthenelse{\equal{\leftmark}{\rightmark}}{}{\rightmark}}
\cfoot*{\pagemark}
\KOMAoptions{headsepline=0.1pt}
\AfterEndEnvironment{figure}{\noindent\ignorespaces}
\AfterEndEnvironment{table}{\noindent\ignorespaces}

\RequirePackage[T1]{fontenc}
\RequirePackage{lmodern}
\RequirePackage{babel}
\RequirePackage{microtype}
\binoppenalty=700
\brokenpenalty=10000
\clubpenalty=10000
\displaywidowpenalty=10000
\exhyphenpenalty=1000
\floatingpenalty=20000
\hyphenpenalty=60
\interlinepenalty=0
\linepenalty=10
\postdisplaypenalty=0
\predisplaypenalty=10000
\relpenalty=500
\widowpenalty=10000
\interfootnotelinepenalty=10000
\frenchspacing    
\RequirePackage[shortcuts]{extdash}

\RequirePackage[
  style=nature_custom,
  citestyle=numeric-comp,
  isbn=false,
  sorting=none,
]{biblatex}
\RequirePackage{csquotes}   % for language specific quotation in literature
% (must be loaded after, e.g., scrreprt)

\RequirePackage{xcolor}
\definecolor{urlcol}{RGB}{0, 102, 204}
\RequirePackage{hyperref}
\hypersetup{colorlinks=true,urlcolor=urlcol,linkcolor=black,citecolor=blue,breaklinks}
\RequirePackage{enumitem}
\setenumerate{listparindent=\parindent,label=(\alph*)}
\RequirePackage{float}
\RequirePackage{booktabs}
\heavyrulewidth=1.1pt

\RequirePackage{mathtools} % has to be loaded before cleveref
\usepackage{amsthm} % load before cleveref for proper referencing
\RequirePackage[english,capitalize]{cleveref}
\newtheoremstyle{standard}{1.0ex}{1.5ex}{\slshape}{}{\bfseries}{}{.7em}{}
\newtheoremstyle{statement}{1.0ex}{1.5ex}{\itshape}{}{\bfseries}{}{.7em}{}
% this has to be defined after cleveref for proper referencing
\theoremstyle{standard}
\newtheorem{definition}{Definition}[subsection]
\newtheorem{remark}[definition]{Remark}
\newtheorem{notation}[definition]{Notation}
\theoremstyle{statement}
\newtheorem{definition_proposition}[definition]{Definition and Proposition}
\newtheorem{lemma}[definition]{Lemma}
\newtheorem{proposition}[definition]{Proposition}
\newtheorem{proposition_definition}[definition]{Proposition and Definition}
\newtheorem{corollary}[definition]{Corollary}
\newtheorem{theorem}[definition]{Theorem}
\makeatletter
\renewenvironment{proof}[1][\proofname]{%
  \par%
  \vspace{-1.9ex}%
  \pushQED{\qed}%
  \normalfont\topsep1\p@\@plus1\p@\relax%
  \trivlist%
  \item[\hskip\labelsep\itshape#1\@addpunct{.}]%
  \ignorespaces%
}{\popQED\endtrivlist\@endpefalse}
\newenvironment{interrupted_proof}[1][\proofname]{%
  \par%
  \pushQED{\qed}%
  \normalfont \topsep1\p@\@plus1\p@\relax%
  \trivlist%
  \item[\hskip\labelsep \itshape #1\@addpunct{.}]%
  \ignorespaces%
}{\popQED\endtrivlist\@endpefalse}
\AfterEndEnvironment{definition}{\@doendpe}
\AfterEndEnvironment{notation}{\@doendpe}
\AfterEndEnvironment{remark}{\@doendpe}
\AfterEndEnvironment{lemma}{\@doendpe}
\AfterEndEnvironment{proposition}{\@doendpe}
\AfterEndEnvironment{proposition_definition}{\@doendpe}
\AfterEndEnvironment{corollary}{\@doendpe}
\AfterEndEnvironment{theorem}{\@doendpe}
\AfterEndEnvironment{proof}{\@doendpe}
\AfterEndEnvironment{interrupted_proof}{\@doendpe}
\makeatother
\crefname{section}{Sec.}{Secs.}%
\crefname{appendix}{App.}{Apps.}%
\crefname{definition}{Def.}{Defs.}%
\crefname{remark}{Rem.}{Rems.}%
\crefname{notation}{Not.}{Nots.}%
\crefname{definition_proposition}{Def.}{Defs.}%
\crefname{lemma}{Lem.}{Lems.}%
\crefname{proposition}{Prop.}{Props.}%
\crefname{proposition_definition}{Prop.}{Props.}%
\crefname{corollary}{Cor.}{Cors.}%
\crefname{theorem}{Thm.}{Thms}%
\Crefname{section}{Section}{Sections}%
\Crefname{appendix}{Appendix}{Appendices}%
\Crefname{definition_proposition}{Definition}{Definitions}%
\Crefname{proposition}{Proposition}{Propositions}%
\Crefname{theorem}{Theorem}{Theorems}%

\RequirePackage{amssymb}
\RequirePackage{siunitx}
\RequirePackage{tensor}
\RequirePackage{icomma}
\RequirePackage{bm}
\usepackage{IEEEtrantools}
\RequirePackage{dsfont}
\RequirePackage{isomath}
\RequirePackage[scr=rsfs]{mathalpha}
\DeclareFontFamily{U}{BOONDOX-cal}{\skewchar\font=45}
\DeclareFontShape{U}{BOONDOX-cal}{m}{n}{
  <-> s*[1.08] BOONDOX-r-cal}{}
\DeclareFontShape{U}{BOONDOX-cal}{b}{n}{
  <-> s*[1.08] BOONDOX-b-cal}{}
\DeclareMathAlphabet{\mathboondox}{U}{BOONDOX-cal}{m}{n}
\SetMathAlphabet{\mathboondox}{bold}{U}{BOONDOX-cal}{b}{n}
\DeclareFontFamily{U}{dutchcal}{\skewchar\font=45}
\DeclareFontShape{U}{dutchcal}{m}{n}{
  <-> \mathalfa@calscaled  dutchcal-r}{}
\DeclareFontShape{U}{dutchcal}{b}{n}{
  <-> \mathalfa@calscaled  dutchcal-b}{}
\DeclareMathAlphabet{\mathdutchcal}{U}{dutchcal}{m}{n}
\SetMathAlphabet{\mathdutchcal}{bold}{U}{dutchcal}{b}{n}

\RequirePackage{xparse}
\RequirePackage{xspace}

% use the following commands, or their shortcuts below, instead of manually settting
% parentheses, braces, ...; they have better typesetting
\DeclarePairedDelimiter\Parenthesis{\lparen}{\rparen}
\DeclarePairedDelimiter\Brackets{\lbrack}{\rbrack}
\DeclarePairedDelimiter\Braces{\lbrace}{\rbrace}
\DeclarePairedDelimiter\AbsoluteValue{\lvert}{\rvert}
\DeclarePairedDelimiter\Norm{\lVert}{\rVert}
\DeclarePairedDelimiter\Bra{\langle}{\rvert}
\DeclarePairedDelimiter\Ket{\lvert}{\rangle}
\DeclarePairedDelimiter\BraKet{\langle}{\rangle}
\DeclarePairedDelimiterX\BraKetTwo[2]{\langle}{\rangle}{%
  #1\delimsize\vert\mathopen{}#2%
}
\DeclarePairedDelimiterX\BraKetThree[3]{\langle}{\rangle}{%
  #1\delimsize\vert\mathopen{}#2%
  \delimsize\vert\mathopen{}#3%
}
\DeclarePairedDelimiterX\BraKetTwoStar[2]{\langle}{\rangle}{%
  #1\nonscript\,\delimsize\vert\nonscript\,\mathopen{}#2}
\DeclarePairedDelimiterX\BraKetThreeStar[3]{\langle}{\rangle}{%
  #1\nonscript\,\delimsize\vert\nonscript\,\mathopen{}#2\nonscript\,%
  \delimsize\vert\nonscript\,\mathopen{}#3%
}
\DeclarePairedDelimiter\DotDot{.}{.}
\DeclarePairedDelimiter\ParenthesisDot{\lparen}{.}
\DeclarePairedDelimiter\DotParenthesis{.}{\rparen}
\DeclarePairedDelimiter\BraceDot{\lbrace}{.}
\DeclarePairedDelimiter\DotBrace{.}{\rbrace}
\DeclarePairedDelimiter\ParenthesisBracket{\lparen}{\rbrack}
\DeclarePairedDelimiter\BracketParenthesis{\lbrack}{\rparen}
\DeclarePairedDelimiterX\SetVerticalLine[2]{\lbrace}{\rbrace}{%
  #1\nonscript\:\delimsize\vert\allowbreak\nonscript\:\mathopen{}#2%
}
\DeclarePairedDelimiterX\SetColon[2]{\lbrace}{\rbrace}{%
  #1\nonscript\::\allowbreak\nonscript\:\mathopen{}#2%
}
\DeclarePairedDelimiterX\GroupSetVerticalLine[2]{\langle}{\rangle}{%
  #1\nonscript\:\delimsize\vert\allowbreak\nonscript\:\mathopen{}#2%
}
\DeclarePairedDelimiterX\GroupSetColon[2]{\langle}{\rangle}{%
  #1\nonscript\::\allowbreak\nonscript\:\mathopen{}#2%
}
% shortcuts
\NewDocumentCommand\pa{sm}{\IfBooleanTF{#1}{\Parenthesis{#2}}{\Parenthesis*{#2}}}
\NewDocumentCommand\bk{sm}{\IfBooleanTF{#1}{\Brackets{#2}}{\Brackets*{#2}}}
\NewDocumentCommand\bc{sm}{\IfBooleanTF{#1}{\Braces{#2}}{\Braces*{#2}}}
\NewDocumentCommand\abs{sm}{\IfBooleanTF{#1}{\AbsoluteValue{#2}}{\AbsoluteValue*{#2}}}
\NewDocumentCommand\norm{sm}{\IfBooleanTF{#1}{\Norm{#2}}{\Norm*{#2}}}
\NewDocumentCommand\dotdot{sm}{\IfBooleanTF{#1}{\DotDot{#2}}{\DotDot*{#2}}}
\NewDocumentCommand\padot{sm}{\IfBooleanTF{#1}{\ParenthesisDot{#2}}{\ParenthesisDot*{#2}}}
\NewDocumentCommand\dotpa{sm}{\IfBooleanTF{#1}{\DotParenthesis{#2}}{\DotParenthesis*{#2}}}
\NewDocumentCommand\bcdot{sm}{\IfBooleanTF{#1}{\BraceDot{#2}}{\BraceDot*{#2}}}
\NewDocumentCommand\dotbc{sm}{\IfBooleanTF{#1}{\DotBrace{#2}}{\DotBrace*{#2}}}
\NewDocumentCommand\bkpa{sm}{\IfBooleanTF{#1}{\BracketParenthesis{#2}}%
  {\BracketParenthesis*{#2}}}
\NewDocumentCommand\pabk{sm}{\IfBooleanTF{#1}{\ParenthesisBracket{#2}}%
  {\ParenthesisBracket*{#2}}}
\NewDocumentCommand\bra{smd""}{%
  \IfBooleanTF{#1}{\Bra{#2}\IfValueTF{#3}{_{#3}}{}}{\Bra*{#2}\IfValueTF{#3}{_{#3}}{}}%
}
\NewDocumentCommand\ket{smd""}{%
  \IfBooleanTF{#1}{\Ket{#2}\IfValueTF{#3}{_{#3}}{}}{\Ket*{#2}\IfValueTF{#3}{_{#3}}{}}%
}
\NewDocumentCommand\braket{smd""ood""d>>}{%
  {}\IfValueTF{#7}{_{\IfValueTF{#3}{#3}{#7}}}{\IfValueTF{#3}{_{#3}}{}}%
  \IfBooleanTF{#1}%
  {%
    \IfValueTF{#4}{\IfValueTF{#5}{\BraKetThree{#2}{#4}{#5}}{\BraKetTwo{#2}{#4}}}%
    {\BraKet{#2}}%
  }%
  {%
    \IfValueTF{#4}{\IfValueTF{#5}{\BraKetThreeStar*{#2}{#4}{#5}}{\BraKetTwoStar*{#2}{#4}}}%
    {\BraKet*{#2}}%
  }%
  \IfValueTF{#7}{_{\IfValueTF{#6}{#6}{#7}}}{\IfValueTF{#6}{_{#6}}{}}%
}
\NewDocumentCommand\ketbra{smd""od""d>>}{%
  \IfBooleanTF{#1}%
  {%
    \Ket{#2}\IfValueTF{#6}{_{\IfValueTF{#3}{#3}{#6}}}%
    {\IfValueTF{#3}{_{#3}}{}}\!\Bra*{\IfValueTF{#4}{#4}{#2}}%
  }%
  {%
    \Ket*{#2}\IfValueTF{#6}{_{\IfValueTF{#3}{#3}{#6}}}%
    {\IfValueTF{#3}{_{#3}}{}}\!\Bra*{\IfValueTF{#4}{#4}{#2}}%
  }%
  \IfValueTF{#6}{_{\IfValueTF{#5}{#5}{#6}}}{\IfValueTF{#5}{_{#5}}{}}%
}
\let\Set\SetVerticalLine % \let\Set\SetColon as alternative
\let\GroupSet\GroupSetVerticalLine
\newcommand\set[2]{\Set*{#1}{#2}}
\newcommand\groupset[2]{\GroupSet*{#1}{#2}}
% subequation and align in one
\newenvironment{subalign}{\subequations\align}{\endalign\endsubequations}
% matrices
\newcommand\mat[1]{\begin{matrix}#1\end{matrix}}
\newcommand\matp[1]{\begin{pmatrix}#1\end{pmatrix}}
\newcommand\matb[1]{\begin{bmatrix}#1\end{vmatrix}}
\newcommand\matB[1]{\begin{Bmatrix}#1\end{Bmatrix}}
\newcommand\matv[1]{\begin{vmatrix}#1\end{vmatrix}}
\newcommand\matV[1]{\begin{Vmatrix}#1\end{Vmatrix}}
\newcommand\smat[1]{\begin{smallmatrix}#1\end{smallmatrix}}
\newcommand\smatp[1]{\Parenthesis*{\begin{smallmatrix}#1\end{smallmatrix}}}

\newcommand\minusset{\!\setminus\!}

% now some specific definitions and shortcuts we are using
\DeclareMathOperator\tr{Tr}
\DeclareMathOperator\id{id}
% Hilbert space, basis, ..
\newcommand*\hi{\mathcal{H}} % hilbert
\newcommand*\ba{\mathcal{B}} % basis
\newcommand*\ls{\mathscr{L}} % linear stetig
% \newcommand*\LL{\mathcal{L}}
\newcommand*\bh{\mathboondox{Q}} % for qubits space / (b)inary (h)ilbert
\newcommand*\de{\mathboondox{D}} % density operator/matrix

% Some common sets, groups, ...
\newcommand*\N{\mathbb{N}}
\newcommand*\No{\mathbb{N}_0}
\newcommand*\Z{\mathbb{Z}}
\newcommand*\Q{\mathbb{Q}}
\newcommand*\R{\mathbb{R}}
\newcommand*\C{\mathbb{C}}
\newcommand*\F{\mathbb{F}}
\newcommand*\K{\mathbb{K}}
\newcommand*\M{\mathrm{M}}
\newcommand*\U{\mathrm{U}}
\newcommand*\SU{\mathrm{SU}}
\newcommand*\GL{\mathrm{GL}}
\newcommand*\Sp{\mathrm{Sp}}

% correct space when writing equations over multiple lines
\newcommand*\eqspace{\mathrel{\phantom{=}}}

% common notations, that may take an argument or not
\NewDocumentCommand\order{o}{\mathcal{O}\IfValueTF{#1}{\pa{#1}}{}}
\let\olddim\dim
\RenewDocumentCommand\dim{o}{\olddim\IfValueTF{#1}{\pa{#1}}{}}
\NewDocumentCommand\aut{m}{\mathrm{Aut}\pa{#1}}
\NewDocumentCommand\cen{o}{\mathrm{Z}\IfValueTF{#1}{\pa{#1}}{}}
\NewDocumentCommand\ces{od""}{\mathrm{C}\IfValueTF{#2}{_{#2}}{}\IfValueTF{#1}{\pa{#1}}{}}
\NewDocumentCommand\nos{od""}{\mathrm{N}\IfValueTF{#2}{_{#2}}{}\IfValueTF{#1}{\pa{#1}}{}}
% \NewDocumentCommand\nos{om}{\mathrm{N}\IfValueTF{#1}{_{#1}}{}\pa{#2}}

% other
\newcommand*\ee{\mathrm{e}}
\let\da\dagger
\newcommand*\ur{^\dagger}
\newcommand\ts{\intercal} % fix that for consistency \newcommand\ts{T}
\newcommand*\ut{^\ts}
\newcommand*\ui{^{-1}}
\newcommand*\inn{\mathrm{inn}}
\newcommand*\1{\mathds{1}}
\newcommand*\E{\mathcal{E}} % "big varepsilon"
\newcommand*\cc{\mathrm{c.c.}}
\newcommand*\hc{\mathrm{h.c.}}

% some specific shortcuts
\NewDocumentCommand\w{sr()}{\IfBooleanTF{#1}{\Parenthesis{#2}}{\Parenthesis*{#2}}}
\let\er\ket
\let\ar\bra
\let\era\ketbra
\let\are\braket
\NewDocumentCommand\en{so}{%
$\IfValueTF{#2}{#2}{n} \in \mathbb{N}\IfBooleanTF{#1}{_0}{}$\xspace%
}
\NewDocumentCommand\hr{mo}{\href{#1}{\IfValueTF{#2}{#2}{#1}}}

% specific quantum computing stuff
\NewDocumentCommand\cz{o}{\mathrm{CZ}\IfValueTF{#1}{_{#1}}{}}
\NewDocumentCommand\cx{o}{\mathrm{CX}\IfValueTF{#1}{_{#1}}{}}
\NewDocumentCommand\cy{o}{\mathrm{CY}\IfValueTF{#1}{_{#1}}{}}
\NewDocumentCommand\zcz{o}{\mathrm{ZCZ}\IfValueTF{#1}{_{#1}}{}}
\NewDocumentCommand\zcx{o}{\mathrm{ZCX}\IfValueTF{#1}{_{#1}}{}}
\NewDocumentCommand\zcy{o}{\mathrm{ZCY}\IfValueTF{#1}{_{#1}}{}}
\NewDocumentCommand\Swap{o}{\mathrm{SWAP}\IfValueTF{#1}{_{#1}}{}}
\NewDocumentCommand\iSwap{o}{\mathrm{iSWAP}\IfValueTF{#1}{_{#1}}{}}
% Pauli and Clifford group
\newcommand*\gp{\mathcal{P}}
\newcommand*\op{\overline{\mathcal{P}}}
\newcommand*\ip{\mathdutchcal{P}}
\newcommand*\uip{\hat{\mathdutchcal{P}}}
\newcommand*\oip{\overline{\mathdutchcal{P}}}
\newcommand*\ouip{\overline{\hat{\mathdutchcal{P}}}}
\newcommand*\gpr{\mathrm{P}}
\newcommand*\gc{\mathcal{C}}
\newcommand*\oc{\overline{\mathcal{C}}}
\newcommand*\ic{\mathdutchcal{C}}
\newcommand*\uic{\hat{\mathdutchcal{C}}}
\newcommand*\oic{\overline{\mathdutchcal{C}}}
\newcommand*\ouic{\overline{\hat{\mathdutchcal{C}}}}
\newcommand*\gcr{\mathrm{C}} % Clifford group representatives
\newcommand*\otau{\overline{\tau}}
